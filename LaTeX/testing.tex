\documentclass[12pt]{report}

% NEEDED FOR CASES
\usepackage{amsmath}

% DOCUMENT SET UP
\title{Title of The Document}
\author{Carson Holland}
\date{\today}
\begin{document}
\maketitle

\tableofcontents

% Chapter 1 --------------------------------------
\chapter{Table}
\label{chapter:table}

This is the chapter table. I a writing more info here.

\section{Section One}
\label{section:one}
This is the first section in Chapter~\ref{chapter:table}. This is Section~\ref{section:one}

\subsection{Subsection 1}
We are talking about something specific within Chapter~\ref{chapter:table}


% TABLE
\begin{table}[h!]
\begin{tabular}{| l | l | l |} \hline
Time (ms) & Meters  & Color  \\ \hline
5.5 ms  & 8meters  & Pink  \\ \hline
6.7 ms & 17 meters & Green \\ \hline
\end{tabular}
\caption{A sample table with ms, meters, and color}
\label{table:my_data}
\end{table}

% Referencing the table later
As shown in Table~\ref{table:my_data}, the values reflect nothing. This data makes no sense.


% Chapter 2 --------------------------------------
\chapter{Math Equations}
\label{chapter:math_equations}

\section{My Used Math Equations}

\begin{enumerate}
    \label{enumerate:equations}
    % Energy Equation
    \item 
        \begin{equation}
        E = mc^2
        \label{equation:energy}
        \end{equation}
    
    % Fake Sin Equation
    \item
        \begin{equation}
        E = sin(2\pi * 10 + t)
        \label{equation:fake_sin}
        \end{equation}

    % Fourier Transform Equation
    \item
        \begin{equation}
            H(s) = \int_{0}^{\infty} x(t) * e^{-2\pi j \omega} \,d\omega
        \label{equation:f_transform}
        \end{equation}
\end{enumerate}
We have a lot of cool stuff in Equation List~\ref{enumerate:equations}. In Equation~\ref{equation:fake_sin} reveals nothing, as this document is just a LaTeX learning tool. That equation is giberish

% Chapter 3 --------------------------------------
\chapter{Items}
\label{chapter:bracket_items}

\section{Brakcet Items Section}
\label{section:b_items}


\[
f(x) = 
\begin{cases}
\label{cases:complex_function}
x & \text{if } x \ge 0 \\
x^3 & \text{if } x < 0 \\
nothing & \text{if } \pi = \infty \\
\end{cases}
\]
As shown in Equation~\ref{cases:complex_function}, that is a piecewise function.

\[
\left[
Sum Algorithm
\begin{array}{l}
\label{array:sum_algorithm} \\
    \text{1. Add up all numbers} \\
    \text{2. Divide the sum from step 1 by the number of items added.} \\
    \text{3. Profit.} \\
    \text{4. Bottom Text} \\
\end{array}
\right]
\]





\end{document}